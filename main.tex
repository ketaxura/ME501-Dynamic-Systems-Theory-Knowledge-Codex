\documentclass{article}
\usepackage{graphicx} % Required for inserting images

\title{ME501 Knowledge Vault}
\author{Usukhbayar Amgalanbat}
\date{October 2025}

\documentclass{article}
\usepackage{amsmath}

\begin{document}

\section*{System Representation}

A dynamic system can be represented in two different ways:  
in \textbf{state-space representation} or in \textbf{transfer function / frequency representation}.

\subsection*{State Representation Types}

State representation itself splits into two different types:

\begin{itemize}
    \item \textbf{LTI} — Linear Time-Invariant
    \item \textbf{LTV} — Linear Time-Varying
\end{itemize}

\subsection*{LTI System}

\begin{equation}
    \dot{x} = A x(t) + B u(t)
\end{equation}

\begin{equation}
    y = Cx
\end{equation}

\subsection*{LTV System}

\begin{equation}
    \dot{x} = A(t)x(t) + B(t)u(t)
\end{equation}

\begin{equation}
    y = C(t)x(t)
\end{equation}

LTI systems do not have their system matrices varying with time.  
That is, \(A, B, C, D\) do not vary with time.

\subsection*{System Matrices \(A, B, C, D\)}

\textbf{A:} Explains how the state vector \(x(t)\) and its changes actually affect the system's future behavior.  
Systems builds on itself and evolves based on its past values over time.  
The \(A\) matrix defines how the past/current state vector influences the future system state and behavior.  
This matrix is crucial for modeling and predicting the system.  
For simple systems without a control input component, matrix \(A\) is the model and prediction mechanism.

\[
A \in \mathbb{R}^{n \times n}
\]



\end{document}
